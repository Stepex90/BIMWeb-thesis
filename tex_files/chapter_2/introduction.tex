In questo capitolo si descrive la scelta fatta per portare la ``filosofia'' del BIM sul Web, tramite il framework sviluppato
\emph{Metior}.
Questo applicativo consente di modellare edifici in ambiente Web, secondo una metodologia che coniughi le potenzialità
dell’approccio BIM con la semplicità di utilizzo.
La semplificazione alla base del progetto sta nel fatto che l'utente non deve interfacciarsi con il classico programma
di CAD con complesse procedure dalla elevata curva di apprendimento, ma, attraverso un approccio parametrizzato,
realizza un modello in modo semplice e veloce.
Si descrive il framework Metior partendo dall'esperienza ad alto livello dell'utente, per scendere in profondità
a basso livello fino a conoscerne l'architettura del framework.
Le funzionalità ``semplificate'' del BIM vengono implementate attraverso l'utilizzo di librerie software open-source,
in particolar modo React~\ref{sec:chapter_2_section_3_sub_1}
 e Threejs~\ref{sec:chapter_2_section_3_sub_2}, dando all'utente un applicativo paragonabile a quelli
Desktop, che consente l'interazione del modello creato tramite le visualizzazione 2D e 3D, sotto le quali
è presente un architettura \emph{serverless}.
\newpage
