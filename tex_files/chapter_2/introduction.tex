In questo capitolo si descrive la scelta fatta per portare la ``filosofia'' del BIM sul Web tramite il framework sviluppato
\emph{Metior}.
Questo applicativo consente di modellare edifici in ambiente Web, secondo una metodologia che coniughi le potenzialità
dell’approccio BIM con la semplicità di utilizzo.
Il nome \emph{Metior} (dal latino: \emph{misurare} o \emph{stimare}) mira alla semplificazione della fase di progettazione,
in modo che l'utente non debba interfacciarsi con il classico programma
di CAD con complesse procedure dalla elevata curva di apprendimento ma, attraverso un approccio parametrizzato,
realizza un modello in modo rapido.
Si descrive il framework\emph{Metior} partendo dall'esperienza ad alto livello dell'utente per poi scendere in profondità
a basso livello fino a conoscerne l'architettura.
Le funzionalità ``semplificate'' del BIM vengono implementate attraverso l'utilizzo di librerie software open-source,
nello specifico \emph{React}~\ref{sec:chapter_2_section_3_sub_1}
 e \emph{Threejs}~\ref{sec:chapter_2_section_3_sub_2}, dando all'utente un applicativo paragonabile a quelli
Desktop, che consente l'interazione del modello creato tramite le visualizzazione 2D e 3D, basato su un'architettura \emph{serverless}.
\newpage
