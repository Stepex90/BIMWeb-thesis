\section{User Experience}
\label{sec:chapter_2_section_5}

The application experience focuses on supporting the user in a building modeling task. The exploited modeling approach requires the user to face as much as possible a two-dimensional interface which allows her to define the plan and to place complex architectural elements (here called \emph{building elements}) on it. Such \emph{building elements} can be found in a pre-filled catalog, and when required can be further configured and customized through a side panel. This modeling approach move part of the complexity toward the developer of the customizable building elements, leaving to the final user the task to place and to configure the employed elements. A rich catalog of elements is thus crucial to answer to the users' modeling requirements.

Once the floor-plan has been defined according to the \emph{place-and-configure} approach, the system can automatically generate a 3D model which can be explored externally or in first person view, as shown in Figure~\ref{fig3D-school}. Each  \emph{building element} in fact comprises either a \emph{2D generating function (2Dgf)} and a \emph{3D generating function (3Dgf)}, used to obtain models used in the 2D floorplan definition and in 3D generated model respectively.

The tool also has support for layers the user can exploit to organize her project, for example to group together semantically homogenous elements.
