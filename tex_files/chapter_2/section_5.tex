\section{User Experience}
\label{sec:chapter_2_section_5}


L'application experience si concentra sul sostegno che l'utente in un compito di modellazione edificio.
L'approccio modellistico sfruttato l'utente deve affrontare il pi\`u possibile un'interfaccia bidimensionale che permette
di definire il piano e per posizionare elementi architettonici complessi (chiamati \emph{building elements}) su di esso.
Tali \emph{building elements} può essere trovato in un catalogo pre-riempito, e quando richiesto può essere
configurato ulteriormente e personalizzato attraverso un pannello laterale. Questa mossa parte approccio
di modellazione della complessità verso lo sviluppatore degli elementi costruttivi personalizzabili,
lasciando all'utente finale il compito di posizionare e configurare gli elementi impiegati.
Un ricco catalogo di elementi è quindi fondamentale per rispondere alle esigenze di modellazione degli utenti.

% The application experience focuses on supporting the user in a building modeling task.
%  The exploited modeling approach requires the user to face as much as possible a two-dimensional interface which allows
%   her to define the plan and to place complex architectural elements (here called \emph{building elements}) on it.
  % Such \emph{building elements} can be found in a pre-filled catalog, and when required can be further configured
  % and customized through a side panel. This modeling approach move part of the complexity toward the developer of
  % the customizable building elements, leaving to the final user the task to place and to configure the employed elements.
  %  A rich catalog of elements is thus crucial to answer to the users' modeling requirements.

Una volta che il flor-plan è stato definito in base al approccio \emph{place-and-configure}, il sistema pu\`o automaticamente
generare un modello 3D che pu\`o essere esplorato esternamente o in prima persona, come mostrato in figura ~\ref{fig3D-school}.
Ogni \emph{building element} infatti comprende sia un \emph{funzione generatrice 2D (2Dgf)} e un
\emph{funzione generatrice 3D (3Dgf)}, utilizzate per ottenere modelli nella planimetria 2D e in 3D che ha generato il modello
rispettivamente.
% Once the floor-plan has been defined according to the \emph{place-and-configure} approach, the system can automatically
%  generate a 3D model which can be explored externally or in first person view, as shown in Figure~\ref{fig3D-school}.
%  Each  \emph{building element} in fact comprises either a \emph{2D generating function (2Dgf)} and a
%  \emph{3D generating function (3Dgf)}, used to obtain models used in the 2D floorplan definition and in 3D generated model
%   respectively.
