\newpage
\section{User Interface}
\label{sec:chapter_2_section_4}

Qui si fa una descrizione della User Interface...

\subsection{Viewer 2D}
Il \emph{2D-viewer} invoca il  \emph{2Dgf} degli elementi costruiti e aggiunti al modello e genera un suo output usando gli elementi SVG
Per far fronte ai frequenti aggiornamenti provenienti dall'interazione con il disegno da parte dell'utente,
sfrutta la \emph{Virtual DOM}~\cite{vdom}, che permette di aggiornare solo la parte modificata evitando così
il completo rendering della scena. Per eseguire le operazioni di pan e zoom, tipicamente necessaria in questo
tipo di strumento, sviluppiamo un componente ad-hoc di React denominato \emph{ReactSVGPanZoom}\footnote{https://github.com/chrvadala/react-svg-pan-zoom}.\\

% The \emph{2D-viewer} invokes the \emph{2Dgf} of the building elements added to the model and renders its output using SVG
% elements. To cope with frequent updates coming from the user drawing interaction, it exploits the \emph{Virtual DOM}~\cite{vdom},
%  which permits to update only the modified part thus avoiding complete redrawing of the scene. To perform pan and zoom operations,
%   typically necessary in this kind of tool, we develop an ad-hoc React component named
  % \emph{ReactSVGPanZoom}\footnote{https://github.com/chrvadala/react-svg-pan-zoom}.

\begin{figure}[htbp] %  figure placement: here, top, bottom, or page
   \centering
   \includegraphics[width=1\linewidth]{images/2d}
   \caption{Schermata viewer 2D}
   \label{fig:view2D}
\end{figure}
\newpage


\subsection{Viewer 3D}
Il \emph{3D-viewer} invoca il \emph{3Dgf} dagli elementi di costruzione aggiungi al model e lo renderizza in output usando le primitive WebGL
\emph{Three.js}~\footnote{https://threejs.org/}. \`E stato implementato un \emph{diff} e \emph{patch} di
sistema, standardizzate in ~\cite{rfc6902}: gli oggetti Three.js sono associati con elementi costruttivi all'interno dello Stato,
in modo che ogni volta che l'utente attiva un'azione che si traduce in una modifica dello stato, l'applicazione calcola
la differenza tra il vecchio stato e quello nuovo e cambia solo l'oggetto interessato. In particolare possiamo avere le seguenti \textit{operazioni}:
(i) \emph{add}, (ii) \emph{replace} and (iii) \emph{remove}.\\
% The \emph{3D-viewer} invokes the \emph{3Dgf} of the building elements added to the model and renders its output using WebGL
%  primitives via \emph{Three.js}~\footnote{https://threejs.org/}. It has been implemented a \emph{diff} and \emph{patch}
%  system, standardized in~\cite{rfc6902}: Three.js objects are associated with building elements inside the state, so every
%   time the user triggers an action that results in a state alteration, the application computes the difference between the
%    old state and the new one and changes only the affected object. In particular we can have the following \textit{operations}:
%     (i) \emph{add}, (ii) \emph{replace} and (iii) \emph{remove}.

\begin{figure}[htbp] %  figure placement: here, top, bottom, or page
   \centering
   \includegraphics[width=1\linewidth]{images/3d}
   \caption{Schermata viewer 3D }
   \label{fig:viewer3D}
\end{figure}
\newpage
