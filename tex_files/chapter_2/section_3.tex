\section{Stack Tecnologico}
\label{sec:chapter_2_section_3}

The UI has been developed following the \emph{Web Components pattern}~\cite{web_components}, supported by \emph{React.js}\footnote{https://github.com/facebook/react} framework. The main idea is to define the frontend application as a collection of independent components, each one referencing a specific subset of the centralized state and able to render itself according to the actual values of that portion of the state. Web Components spawn from for high level generic containers, like the toolbar or the catalog, to very fine grained ones, buttons for example. The most interesting are the viewers of the building model: the \emph{2D-viewer} and the \emph{3D-viewer}.

\subsection{React}
React (sometimes styled React.js or ReactJS) is an open-source JavaScript library for building user interfaces.

It is maintained by Facebook, Instagram and a community of individual developers and corporations.[2][3][4] According to JavaScript analytics service Libscore, React is currently being used on the websites of Netflix, Imgur, Bleacher Report, Feedly, Airbnb, SeatGeek, HelloSign, Walmart, and others.[5]

\subsubsection{history}
React was created by Jordan Walke, a software engineer at Facebook. He was influenced by XHP, an HTML component framework for PHP.[6] It was first deployed on Facebook's newsfeed in 2011 and later on Instagram.com in 2012.[7] It was open-sourced at JSConf US in May 2013. React Native, which enables native iOS, Android and UWP development with React, was announced at Facebook's React.js Conf in February 2015 and open-sourced in March 2015.

\subsubsection{Notable features}
One-way data flow[edit]
Properties, a set of immutable values, are passed to a component's renderer as properties in its HTML tag. A component cannot directly modify any properties passed to it, but can be passed callback functions that do modify values. This mechanism's promise is expressed as "properties flow down; actions flow up".

\subsubsection{Virtual DOM}
Another notable feature is the use of a "virtual Document Object Model," or "virtual DOM." React creates an in-memory data structure cache, computes the resulting differences, and then updates the browser's displayed DOM efficiently.[8] This allows the programmer to write code as if the entire page is rendered on each change while the React libraries only render subcomponents that actually change.

\subsubsection{JSX}
React components are typically written in JSX, a JavaScript extension syntax allowing quoting of HTML and using HTML tag syntax to render subcomponents.[9] HTML syntax is processed into JavaScript calls of the React library. Developers may also write in pure JavaScript. JSX is similar to another extension syntax created by Facebook for PHP, XHP.

\subsubsection{Architecture beyond HTML}
The basic architecture of React applies beyond rendering HTML in the browser. For example, Facebook has dynamic charts that render to <canvas> tags,[10] and Netflix and PayPal use isomorphic loading to render identical HTML on both the server and client.[11][12]

\subsubsection{React Native}
See also: WinJS
React Native libraries were announced by Facebook in 2015,[13] providing the React architecture to native iOS, Android and UWP[14] applications.

\newpage
\subsection{Threejs}
Three.js is a cross-browser JavaScript library/API used to create and display animated 3D computer graphics in a web browser. Three.js uses WebGL. The source code is hosted in a repository on GitHub.

\subsubsection{Overview}
Three.js allows the creation of GPU-accelerated 3D animations using the JavaScript language as part of a website without relying on proprietary browser plugins.[3][4] This is possible thanks to the advent of WebGL.[5]

High-level libraries such as Three.js or GLGE, SceneJS, PhiloGL or a number of other libraries make it possible to author complex 3D computer animations that display in the browser without the effort required for a traditional standalone application or a plugin.[6]

\subsubsection{History}
Three.js was first released by Ricardo Cabello to GitHub in April 2010.[2] The origins of the library can be traced back to his involvement with the demoscene in the early 2000s. The code was first developed in ActionScript, then in 2009 ported to JavaScript. In Cabello's mind, the two strong points for the transfer to JavaScript were not having to compile the code before each run and platform independence. With the advent of WebGL, Paul Brunt was able to add the renderer for this quite easily as Three.js was designed with the rendering code as a module rather than in the core itself.[7] Cabello's contributions include API design, CanvasRenderer, SVGRenderer and being responsible for merging the commits by the various contributors into the project.

The second contributor in terms of commits, Branislav Ulicny started with Three.js in 2010 after having posted a number of WebGL demos on his own site. He wanted WebGL renderer capabilities in Three.js to exceed those of CanvasRenderer or SVGRenderer.[7] His major contributions generally involve materials, shaders and post-processing.

Soon after the introduction of WebGL on Firefox 4 in March 2011, Joshua Koo came on board. He built his first Three.js demo for 3D text in September 2011.[7] His contributions frequently relate to geometry generation.

There are over 650 contributors in total.[7]

\subsubsection{Features}
Three.js includes the following features:[8]

Effects: Anaglyph, cross-eyed and parallax barrier.
Scenes: add and remove objects at run-time; fog
Cameras: perspective and orthographic; controllers: trackball, FPS, path and more
Animation: armatures, forward kinematics, inverse kinematics, morph and keyframe
Lights: ambient, direction, point and spot lights; shadows: cast and receive
Materials: Lambert, Phong, smooth shading, textures and more
Shaders: access to full OpenGL Shading Language (GLSL) capabilities: lens flare, depth pass and extensive post-processing library
Objects: meshes, particles, sprites, lines, ribbons, bones and more - all with Level of detail
Geometry: plane, cube, sphere, torus, 3D text and more; modifiers: lathe, extrude and tube
Data loaders: binary, image, JSON and scene
Utilities: full set of time and 3D math functions including frustum, matrix, quaternion, UVs and more
Export and import: utilities to create Three.js-compatible JSON files from within: Blender, openCTM, FBX, Max, and OBJ
Support: API documentation is under construction, public forum and wiki in full operation
Examples: Over 150 files of coding examples plus fonts, models, textures, sounds and other support files
Debugging: Stats.js,[9] WebGL Inspector,[10] Three.js Inspector[11]
Three.js runs in all browsers supported by WebGL.

Three.js is made available under the MIT license.[1]
