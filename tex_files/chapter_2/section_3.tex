\section{Stack Tecnologico}
\label{sec:chapter_2_section_3}

La User Interface è stata sviluppata seguendo i \emph{Web Components pattern}~\cite{web_components},
supportati dal framework \emph{React}.
L'idea principale è definire un applicazione di frontend come una collezione di componenti indipendenti,
ognuno dei quali si riferisce ad uno specifico sottoinsieme di stati centralizzati ed in grado
di fare il render in accordo con i valori effettivi di quella porzione dello stato.
I Web Components sono la base per contenitori generici di alto livello, come la barra degli strumenti o il catalogo,
a di quelli a grana molto fine, come ad esempio i pulsanti. I più interessanti sono i visualizzatori del
modello di costruzione: il \emph{2D-viewer} e \emph{3D-viewer}.

\subsection{React}
\label{sec:chapter_2_section_3_sub_1}
React (definito React.js or ReactJS) è una libreria open-source JavaScript per la costruzione di una interfaccia utente.
\`E sviluppato da Facebook e sostenuto da Instagram e da una comunità di sviluppatori tramite la sua disponibilità su repository GitHub
~\cite{infoworld}~\cite{facebookreact}~\cite{reactjs}. Secondo il servizio di analisi JavaScript Libscore~\cite{libscope},
React è attualmente utilizzato sui  siti web di Netflix, Imgur, Bleacher Report, Feedly, Airbnb, SeatGeek,
HelloSign, Walmart, e altri.

\subsubsection{Storia}
React è stato creato da Jordan Walke, un ingegnere del software di Facebook. \`E stato influenzato da XHP, un HTML
framework di componenti per PHP. \`E stato distribuito sulle newsfeed di Facebook nel 2011 e in seguito
Instagram.com nel 2012. \`E stato reso open-source durante il JSConf Stati Uniti nel maggio 2013.
React Native, che consente lo sviluppo di applicazioni native su iOS, Android e Windows Mobile attraverso React,
è stato annunciato alla Facebook's React.js Conf in febbraio 2015 e reso open-source nel marzo 2015.

\subsubsection{Flusso dati unidirezionale}
Le propriet\`a ed un insieme di valori immutabili sono passate al componente di rendering come propriet\`a nel suo tag HTML.
Un componente non pu\`o modificare direttamente le propriet\`a che gli sono state passate, ma pu\`o usare funzioni di
callback le quali vanno a modificare i valori.

\subsubsection{Virtual DOM}
Un'altra caratteristica degna di nota è l'utilizzo di un "virtual Document Object Model", o "virtual DOM".
React crea una struttura cache dei dati in memoria, calcola le differenze risultanti, e poi aggiorna
la DOM~\cite{reactdom} del browser visualizzandola in maniera efficiente.
Questo permette al programmatore di scrivere codice come se
l'intera pagina venisse cambiata, mentre le librerie React fanno il render delle sole sottocomponenti che in realtà cambiano.

\subsubsection{JSX}
I componenti React sono tipicamente scritti in JSX, una estensione della sintassi JavaScript che permette di citare
HTML e utilizzano la sintassi dei tag HTML per fare  il render delle sottocomponenti.~\cite{jsx}
La sintassi HTML è trasformata in chiamate JavaScript
della libreria React. Gli sviluppatori possono anche scrivere in JavaScript puro.


\newpage
\subsection{Threejs}
\label{sec:chapter_2_section_3_sub_2}
Three.js \`e una libreria cross-browser JavaScript utilizzata per creare e visualizzare grafica animata in 3D
in un browser web. Three.js utilizza WebGL. Il codice sorgente è ospitato in un repository su GitHub~\cite{github}.

\subsubsection{Overview}
Three.js permette la creazione di animazioni 3D accelerate dalla GPU utilizzando il linguaggio JavaScript
come parte di un sito web senza fare affidamento su Plugin proprietari dei browser.~\cite{O3D}~\cite{unity}
 Questo è possibile grazie all'avvento delle specifiche WebGL.~\cite{khronos}

\subsubsection{Storia}
Three.js è stato pubblicato da Riccardo Cabello su GitHub nel mese di aprile 2010.~\cite{Firstcommit}
Le origini della libreria risalgono al coinvolgimento di Cabello con il demoscene nei primi anni 2000.
Il codice \`e stato sviluppato originariamente in ActionScript, poi nel 2009 portato su JavaScript per l'ovvio vantaggio della
portabilità si ogni piattaforma. Con l'avvento di WebGL, Paul Brunt è stato in grado di integrare il renderer
WebGL all'interno di Three.js creando un modulo di rendering appropriato.~\cite{develop}
I contributi di Cabello comprendono la progettazione API, CanvasRenderer, SVGRenderer ed è il
responsabile per la gestione dei commit da parte dei vari collaboratori nel progetto.

Il secondo contributo, in termini di commit, è di Branislav Ulicny che ha iniziato a lavorare su Three.js nel 2010
dopo aver pubblicato una serie di demo WebGL sul proprio sito. Voleva che la capacit\`a di rendering di WebGL in Three.js
fossero superiori a quelle di CanvasRenderer o SVGRenderer.~\cite{develop}\\
I suoi maggiori contributi comportano migliorie nei moduli dei materiali, degli shaders e di post-processing.

\subsubsection{Caratteristiche}
Three.js include le seguenti caratteristiche:~\cite{mrdoob}
\begin{itemize}

\item Effetti: Anaglyph, cross-eyed e parallax barrier;
\item Scena: aggiunta e rimozione oggetti a run-time;
\item Camera: prospettica e ortogonale; controllers: trackball, FPS, path e molte altre;
\item Animazioni: armatures, forward kinematics, kinematics inversa, morph e keyframe;
\item Luci: ambientale, direzionale, in un punto e spot lights; shadows: generata e ricevuta;
\item Materiali: Lambert, Phong, smooth shading, textures e molte altre;
\item Shaders: l'accesso alle capacità OpenGL Shading Language(GLSL) : lens flare, passaggio in profondità e un vasta libreria di post-elaborazione;
\item Oggetti: meshes, particles, sprites, lines, ribbons, bones and more - tutti con un livello di dettaglio;
\item Geometrie: piani, cubi, sfere, tori, testo 3D e molte altre; modificatori: lathe, extrude e tube;
\item Data loaders: binary, image, JSON e scene;
\item Utilities: set completo di funzioni matematiche in 3D tra cui troncoconiche, matrici, quaternion, UVs e molte altre
\item Export and import: utilities per creare file JSON-Three.js compatibili con : Blender, openCTM, FBX, Max, e OBJ
\item Supporto: documentazione delle API è in costruzione, forum pubblico e wikipedia;
\item Esempi: Oltre 150 file di codice d'esempio con diversi tipi di carattere, modelli, texture, suoni e altri file di supporto:
\item Debugging: Stats.js,~\cite{stats.js} WebGL Inspector,~\cite{webglinspector} Three.js Inspector~\cite{threejsinspector}.
\end{itemize}

Three.js è eseguibile in tutti i browsers che supportano WebGL.
