\section{Propriet\`a Specifiche}
\label{sec:chapter_3_section_3}

\noindent

Ogni plugin ha una serie di propriet\`a specifiche degli elementi costruttivi che rappresenta.
Ogni propriet\`a \`e definita da:
\begin{itemize}
  \item un \emph{name};
  \item un \emph{type} (come ``number'', ``text'', ``boolean'', o ``custom'');
  \item un \emph{value}.
\end{itemize}
In accordo con il proprio tipo, ciascun valore di proprietà può essere inserito in diversi modi.
Ad esempio, un valore della proprietà booleana è impostato tramite una casella di controllo,
mentre una proprietà di testo è impostata attraverso una casella di testo.

Il sistema \`e progettato per accettare tipi di propriet\`a custom. Un propriet\`a custom \'e richiesta per definire
il componente della UI che permette allo user di inserire il suo valore.
Per esempio, una propriet\`a ``colore'' pu\`o essere introdotta definendo un componente della UI composto da tre box di testo
(ad esempio per ogni componente RGB), mentre una propriet\`a ``length'' pu\`o essere introdotta definendo un componente UI
includendo una box di testo per il valore e menu drop-down per le unit\'a di misura.

Le propriet\`a specifiche di un elemento possono essere modificate nel relativo pannello nella sidebar, una volta che l'elemento
\`e stato selezionato nel canvas.
\newpage

% Each plugin has a set of specific properties of the building elements it represents.
% Each property is defined by (1) a \emph{name}, (2) a \emph{type}, such as ``number'', ``text'', ``boolean'', or ``custom'', and by  (3) a \emph{value}.
% According to its type, each property value can be inserted in different ways.
% For example, a boolean property value is set through a checkbox, while a textual property is set through a text box.

% The system is designed to accept custom kinds of property. A custom property is required to define the component of the UI
%  that permits the user to insert its value.
% For example, a ``color'' property can be introduced by defining a UI component composed by three text boxes
% (one for each RGB components), while a ``length'' property can be introduced by defining a UI component including a text box
%  for the value and a drop-down menu for the unit of measure.

% The specific properties of an element can be edited in the relative panel in the sidebar, once the element is selected in the canvas.
