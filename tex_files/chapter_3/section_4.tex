\section{Generazione modelli server-side}
\label{sec:chapter_3_section_4}

\noindent
La generazione dei modelli 3D e 2D è realizzata in modalità \emph{asynchronous}.
Il ritorno dell'invocazione di una funzione generatrice non \`e il modello stesso,
ma una \emph{promise}\footnote{\url{https://www.promisejs.org/}} che sarà risolta con l'output del rendering.
Tale scelta progettuale \`e importante poich\'e il calcolo per la
generazione di modello pu\`o richiedere una notevole quantità di tempo.
Nel frattempo l'utente deve essere in grado di interagire con l'interfaccia, che a sua volta deve rimanere reattiva.
Basandosi su questa architettura, la generazione dei modelli può essere facilmente delegata a un server,
sollevando così il browser dall'onere di calcoli computazionalmente pesanti. Il server espone un API JSON REST-like.

% I plugin possono essere eseguiti si da un client che da server, dal momento che le funzioni generatrici 2D e 3D (\emph{2Dgf} e
% \emph{3Dgf}) definite dai plugin possono essere eseguite dal server.(RIVEDERE ultima frase codice isomorfo)

\newpage
