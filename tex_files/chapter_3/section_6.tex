\section{Server Framework API}
\label{sec:chapter_3_section_6}

\noindent
Our building deconstruction framework  has a web-based client-server architecture,  discussed in Section~\ref{sec:architecture}.  \emph{Metior}, the web client application, is illustrated in Section~\ref{sec:application}. The server-side of the framework, discussed in this section, is a  plugin server written in Python, which capitalizes on the stack of geometric programming tools described above.

The Metior user quickly develops a 3D hierarchical assembly of different parts of the building envelope, as well as the horizontal and vertical partitions, using very simple 2D drawing tools. The more geometrically complex parts of the construction are conversely set up by user picking from context-based boards of predefined plugin templates, that are Python scripts~(see Figure~\ref{spiralstair}) generating solids models which are interactively dimensioned, either using 2D drawing tools, or by user's numeric input from keyboard.

Of course, our list of \emph{plugin templates} embraces most of building parts that are not manageable for quick shape input via 2D interaction. In particular, the picking boards include templates for planar concrete frames, spatial building frames, building foundations, roofs and stairs of different types, attics and dormers, fireplaces and fitted wardrobes, shover cabins and sanitary equipments, doors and windows, etc.

It is worth noting that, by virtue of the great expressiveness of the PLaSM operators and its functional style of programming and dimension-independent geometry, the development of a new plugin template is very easy even for non-experienced programmers, and usually requires a tiny amount of time and code, that may range between 4-8 hours, and between 10-100 lines of Python/pyplasm code.

Two important points we would like to remark are: (a) the great \emph{expressive power} of the geometric language,  strongly empowered by  currying, i.e.~by translating the evaluation of a function---that takes either multiple arguments or a tuple of arguments---into evaluating a sequence of functions, each with a single argument; (b) the \emph{ease of development}. Python/pyplasm is used even to teach geometric programming to K12 students~\cite{ncLab} (see \href{https://nclab.com/3d-gallery/}{\texttt{https://nclab.com/3d-gallery/}}).
Several plugin templates used by Metior were developed in class by students, in the framework of the computer  graphics course being taught by one of authors.
