\section{Tassonomia}
\label{sec:chapter_3_section_2}

\noindent
I \emph{Plugin} possono essere organizzati in accordo con \emph{occupation type} e \emph{placement type}.
L'\emph{occupation type} può essere identificato da tre differenti tipi di Plugin:
\begin{itemize}
  \item \emph{linear};
  \item \emph{area};
  \item \emph{volume};
\end{itemize}
Quello \emph{linear} si estende in una dimensione (a meno di uno spessore radiale) (e.g. linee idrauliche, cavi elettrici).
Il Plugin \emph{area} si estende in due dimensioni (a meno di uno spessore lineare) (e.g. elementi di separazione).
Si possono dividere in \emph{horizontal area} (e.g. pavimento e celle), e \emph{vertical area}, (e.g. muri).
Il Plugin \emph{volume} si estende in tre dimensioni. Si possono avere \emph{fixed volume}, (e.g. un pezzo di arredo) e
un \emph{scalable volume}, che può essere scalato proporzionalmente o no, (e.g. pilastri, scale).


L'\emph{occupation type} determina un modo differente di instanziare e inserire i Plugin nel canvas.
In particolare, nel \emph{2D-mode}, i Plugin \emph{linear} sono inseriti disegnando linee attraverso l'interazione drag\&drop;
Il Plugin \emph{area} sono inseriti disegnando una bounding-box dell'elemento attraverso l'interazione drag\&drop;
Il Plugin \emph{volume} sono inseriti scegliendo la posizione dell'elemento attraverso l'interazione point\&click,
e sistemando la loro dimensione modificando la bounding-box attraverso il drag\&drop.


Il \emph{placement type} determina se l'elemento può essere inserito all'interno del canvas in un specifico punto occupato o meno
da altri elementi. In altre parole, esso determina la relazione tra una nuova instanza del Plugin e l'instanza di altri
Plugin precedentemente aggiunti al modello. La relazione può essere di due tipi: \emph{inside} o \emph{over}.
I Plugin appartenenti alla categoria \emph{inside} posso essere aggiunti solo all'interno di altri elementi (che possono essere
\emph{linear}, \emph{area} o \emph{volume}); e.g., una ``finestra'' è un elemento ``volume inside vertical area'',
mentre un ``linea idraulica'' \`e un elemento ``linear inside horizontal area''.
I Plugin della categoria \emph{over} possono essere aggiunti solo sopra ad altri elementi (di qualsiasi tipo)
e.g., un ``pilastro'' \`e un elemento ``volume over horizontal area'',
mentre un ``pannello elettrico'' è un elemento``volume over vertical area''.
In fase di progettazione, un elemento che non soddisfa i vincoli di posizionamento definiti dal \emph{placement type} \`e
notificato dal sistema come un warning, visualizzando la sua bounding-box in semitrasparenza di colore rosso.
\newpage
