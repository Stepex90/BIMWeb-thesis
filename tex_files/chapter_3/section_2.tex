\section{Tassonomia}
\label{sec:chapter_3_section_2}

\noindent
The plugins can be organized according to \emph{occupation type} and \emph{placement type}.

In the \emph{occupation type} three different kind of plugins can be identified: \emph{linear}, \emph{area} or \emph{volume} plugins.
The \emph{linear} ones extend in one dimension (unless a radial thickness) (e.g. hydraulic lines, electrical cables). The \emph{area} plugins extend in two dimensions (unless a linear thickness), (e.g. separation elements). They can be divided into \emph{horizontal area} (e.g. floor and ceil), and \emph{vertical area}, (e.g. walls). The \emph{volume} plugins extend in three dimensions. They can be \emph{fixed volume}, (e.g. a piece of furniture) and \emph{scalable volume}, that can be scaled (proportionally or not), (eg. pillars, staircases).

The \emph{occupation type} determines a different way to instantiate and to insert the plugins into the canvas.
In particular, in \emph{2D-mode}, \emph{linear} plugins are inserted drawing lines by mean of a drag\&drop interaction;
the \emph{area} plugins are inserted drawing the bounding-box of the element by mean of a drag\&drop interaction;
the \emph{volume} plugins are inserted picking the position of the element by mean of a point\&click interaction,
and adjusting their dimensions modifying the bounding-box by drag\&drop.

The \emph{placement type} determines if the element can be inserted into the canvas in a specific point occupied or not by other elements. In other words, the {placement type} determines the relationship between a new instance of a plugin and instances of other plugins previously added to the model. The relationship can be of two kind: \emph{inside} or \emph{over}.
Plugins belonging to the \emph{inside} category can be added only inside other element (that can be \emph{linear}, \emph{area} or \emph{volume}); e.g., a ``window'' is a ``volume inside vertical area'' element,
while an ``hydraulic line'' is a ``linear inside horizontal area'' element.
Plugins of the \emph{over} category can be added only over other elements (of any type);
e.g., a ``pillar'' is a ``volume over horizontal area'' element,
while an ``electric panel'' is a ``volume over vertical area'' element.

In the design phase, an element that doesn't meet the placement constraints defined by the \emph{placement type} is notified by the system as a visual warning, showing its bounding-box in semi-transparent blinked red color.
