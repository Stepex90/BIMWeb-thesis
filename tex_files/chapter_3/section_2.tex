\section{Tassonomia}
\label{sec:chapter_3_section_2}

\noindent
I plugins posso essere organizzati in accordo con \emph{occupation type} e \emph{placement type}.
L'\emph{occupation type} pu\'o essere identificato da tre differenti tipi di plugins:
\begin{itemize}
  \item \emph{linear};
  \item \emph{area};
  \item \emph{volume};
\end{itemize}
Quello \emph{linear} si estende in una dimensione (a meno di uno spessore radiale) (e.g. linee idrauliche, cavi elettrici).
Il plugin \emph{area} si estende in due dimensioni (a meno di uno spessore lineare) (e.g. elementi di separazione).
Si possono dividere in \emph{horizontal area} (e.g. pavimento e celle), e \emph{vertical area}, (e.g. muri).
Il plugin \emph{volume} si estende in tre dimensioni. Si possono avere \emph{fixed volume}, (e.g. un pezzo di arredo) e
un \emph{scalable volume}, che pu\'o essere scalato (proporzionalmente o no), (e.g. pilastri, scale).

% The plugins can be organized according to \emph{occupation type} and \emph{placement type}.
% In the \emph{occupation type} three different kind of plugins can be identified: \emph{linear}, \emph{area} or \emph{volume} plugins.
% The \emph{linear} ones extend in one dimension (unless a radial thickness) (e.g. hydraulic lines, electrical cables).
% The \emph{area} plugins extend in two dimensions (unless a linear thickness), (e.g. separation elements).
% They can be divided into \emph{horizontal area} (e.g. floor and ceil), and \emph{vertical area}, (e.g. walls).
% The \emph{volume} plugins extend in three dimensions. They can be \emph{fixed volume}, (e.g. a piece of furniture) and
%   \emph{scalable volume}, that can be scaled (proportionally or not), (eg. pillars, staircases).

L'\emph{occupation type} determina un modo differente di instanziare e inserire i plugin nel canvas.
In particolare, nel \emph{2D-mode}, i plugins \emph{linear} sono inseriti disegnando linee attraverso l'interazione drag\&drop;
Il plugins \emph{area} sono inseriti disegnando una bounding-box dell'elemento attraverso l'interazione drag\&drop;
Il plugins \emph{volume} sono inseriti scegliendo la posizione dell'elemento attraverso l'interazione point\&click,
e sistemando la loro dimensione modificando la bounding-box attraverso il drag\&drop.
% The \emph{occupation type} determines a different way to instantiate and to insert the plugins into the canvas.
% In particular, in \emph{2D-mode}, \emph{linear} plugins are inserted drawing lines by mean of a drag\&drop interaction;
% the \emph{area} plugins are inserted drawing the bounding-box of the element by mean of a drag\&drop interaction;
% the \emph{volume} plugins are inserted picking the position of the element by mean of a point\&click interaction,
% and adjusting their dimensions modifying the bounding-box by drag\&drop.

Il \emph{placement type} determina se l'elemento pu\'o essere inserito all'interno del canvas in un specifico punto occupato o meno
da altri elementi. In altre parole, esso determina la relazione tra una nuova instanza del plugin e l'instanza di altri
plugins precedentemente aggiunti al modello. La relazione pu\'o essere di due tipi: \emph{inside} o \emph{over}.
I plugins appartenenti alla categoria \emph{inside} posso essere aggiunti solo all'interno di altri elementi (che possono essere
\emph{linear}, \emph{area} o \emph{volume}); e.g., una ``finestra'' \'e un elemento ``volume inside vertical area'',
mentre un ``linea idraulica'' \`e un elemento ``linear inside horizontal area''.
I plugins della categoria \emph{over} possono essere aggiunti solo sopra ad altri elementi (di qualsiasi tipo)
e.g., un ``pilastro'' \`e un elemento ``volume over horizontal area'',
mentre un ``pannello elettrico'' \'e un elemento``volume over vertical area''.
In fase di progettazione, un elemento che non soddisfa i vincoli di posizionamento definiti dal \emph{placement type} \`e
notificato dal sistema come un warning, visualizzando la sua bounding-box in semitrasparenza di colore rosso.
% The \emph{placement type} determines if the element can be inserted into the canvas in a specific point occupied or not by
% other elements. In other words, the {placement type} determines the relationship between a new instance of a plugin and
% instances of other plugins previously added to the model. The relationship can be of two kind: \emph{inside} or \emph{over}.
% Plugins belonging to the \emph{inside} category can be added only inside other element (that can be \emph{linear}, \emph{area}
% or \emph{volume}); e.g., a ``window'' is a ``volume inside vertical area'' element,
% while an ``hydraulic line'' is a ``linear inside horizontal area'' element.
% Plugins of the \emph{over} category can be added only over other elements (of any type);
% e.g., a ``pillar'' is a ``volume over horizontal area'' element,
% while an ``electric panel'' is a ``volume over vertical area'' element.
% In the design phase, an element that doesn't meet the placement constraints defined by the \emph{placement type} is notified
% by the system as a visual warning, showing its bounding-box in semi-transparent blinked red color.
