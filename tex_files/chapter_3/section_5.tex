\section{Server-side models generation}
\label{sec:chapter_3_section_5}

\noindent
Tra i modelli 3D e 2D generati abbiamo progettato un \emph{asynchronous}.
Il risultato attuale dell'invocazione di una funzione generatrice non \`e generare il modello stesso, ma inoltre una \emph{promise}
di un risultato previsto. Tale scelta progettuale \`e importante poich\'e il calcolo per la generazione di modello pu\`o richiedere
un certo tempo.
Nel frattempo l'utente deve essere in grado di interagire con l'interfaccia, che a sua volta deve rimanere reattiva.
Basandosi su questa architettura, la generazione dei modelli pu\'o essere facilmente delegata a un server,
sollevando così il cliente dall'onere di calcoli onerosi.  Il server espone un REST-like HTTP-based JSON API al cliente.
I plugin spaziano dal client al server, dal momento che le funzioni generatrici 2D e 3D (\emph{2Dgf} e \emph{3Dgf})
 definito dal plugin sono effettivamente eseguite sul server.

% Both the 3D and 2D model generations have been designed as \emph{asynchronous}.
%  The actual result of the invocation of a generating function is not the generated model itself, but rather a \emph{promise}
%   of the expected result. Such a design choice is important since the computation for model generation may require some while.
%    In the meantime the user must be able to interact with the interface, which in turn must remain responsive.
%    Relying on this architecture, generation of the models can be easily delegated to a server
%    (as shown in Figure~\ref{fig:c-s-arch}), thus relieving the client from the burden of onerous computations.
%     The server exposes a REST-like HTTP-based JSON API to the client. The plugins span from the client to the server,
%      since the 2D and 3D generating functions ( \emph{2Dgf} and \emph{3Dgf}) defined by the plugin are actually executed
%       on the server, as shown in Figure~\ref{fig:c-s-arch}.
