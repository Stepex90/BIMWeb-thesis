\section{Processo implementativo}
\label{sec:chapter_3_section_5}

Con processo implementativo di un \emph{Plugin} si intende la formalizzazione delle fasi progettuali.
La formalizzazione del procedimento traccia metodologicamente una sequenza di passi da seguire:
(a) studio della geometria, (b) definizione della funzione di render 2D, (c) definizione della funzione di render 3D,
 (d) instanziazzione nel catalogo, (e) testing nella scena.


\subsection{Studio della geometria}
Nella prima fase implementativa dopo aver scelto quale oggetto andare a modellare all'interno del framework,
si passa ad uno studio della geometria dell'oggetto reale (Figura \ref{fig:confronto} - a), soprattutto in presenza di un modello complesso bisogna
pensare a come realizzarlo cercando di scinderlo in parti geometriche semplici. Ad esempio prendiamo il caso
di un estintore, esso è un oggetto composto da una geometria complessa, ed è scindibile in parti geometriche semplice
come un cilindro più una sfera per il corpo centrale (Figura \ref{fig:confronto} - b) e si segue questa procedura
per tutte le componenti dell'oggetto.\\

   \begin{figure}[htbp]
   \begin{center}
   \begin{tabular}{lcr @{\hspace{2cm}} lr}
   \includegraphics[width=3.8cm]{images/estintore2} &
   \includegraphics[width=3.8cm]{images/estintore} &
   \includegraphics[width=3.8cm]{images/estintore}\\
    (a) & (b) & (c)\\
   \end{tabular}
   \end{center}
   \caption{(a) Foto modello reale, (b) Modello 3D (c) Modello 3D esploso in elementi geometri semplici}
   \label{fig:confronto}
   \end{figure}

Dopo aver scompoto l'oggetto in elementi geometrici semplici, è necessario reperire le informazioni come
larghezza, profondità, altezza e altri dati utili alla modellazione.


\subsection{Definizione funzione 2D}
La funzione di render 2D, attraverso il linguaggio SVG (Scalable Vector Graphics),
consente di visualizzare oggetti di grafica vettoriale e, pertanto, di gestire immagini scalabili dimensionalmente.
Questa definizione consente di inserire all'interno del Virtual DOM della content-area la rappresentazione 2D dell'oggetto
aggiornando l'area interessata del canvas.
In questo modo il Plugin viene visualizzato all'interno di Metior durante la visualizzazione in 2D della scena.\\


\lstset{
    basicstyle=\fontfamily{cr}\selectfont\footnotesize\color{black},
    numbers=none, % where to put the line-numbers
    numberstyle=\tiny, % the size of the fonts that are used for the line-numbers
    backgroundcolor=\color{white},
    showspaces=false, % show spaces adding particular underscores
    showstringspaces=false, % underline spaces within strings
    showtabs=false, % show tabs within strings adding particular underscores
    frame=single, % adds a frame around the code
    tabsize=2, % sets default tabsize to 2 spaces
    captionpos=b, % sets the caption-position to bottom
    breaklines=true, % sets automatic line breaking
    breakatwhitespace=false,
    xleftmargin=17pt,
    framexleftmargin=17pt,
    framexrightmargin=17pt,
    framexbottommargin=5pt,
    framextopmargin=5pt
}
\begin{lstlisting}[
                   numbers=left,
                   frame=trBL,
                   basicstyle=\tiny,
                   caption={Definizion della funzione di render 2D},
                   label=struttura]
   render2D: (element, layer, scene) => {
      let style = {
        stroke: "#000",
        strokeWidth: element.selected ? "2px" : "0px",
        fill: element.properties.get('color')
      };

      return (
        <g transform='translate(-50, -50)'>
          <rect x="0" y="0" width="100" height="100" style={style}/>
        </g>
      );
    },
\end{lstlisting}
\newpage

\subsection{Definizione funzione 3D}
La funzione di render 3D definisce attraverso una porzione di codice scritto in Javascript ed utilizzando
la libreria open-source Threejs, un modello 3D del Plugin;
questa definizione ne consente pertanto la rappresentazione all'interno di Metior durante la visualizzazione in 3D della scena.\\


\begin{lstlisting}[
                   numbers=left,
                   frame=trBL,
                   basicstyle=\tiny,
                   caption={Definizione della funzione render 3D},
                   label=struttura]
 render3D: (element, layer, scene) => {
   let geometry = new BoxGeometry(100, 100, 100);
   let material = new MeshBasicMaterial({
     color: element.properties.get('color')
   });

   let mesh = new Mesh(geometry, material);

   if (element.selected) {
     let box = new BoxHelper(mesh, '#000000');
     box.material.linewidth = 1;
     box.material.depthTest = false;
     box.renderOrder = 1000;
     mesh.add(box);
   }

   mesh.position.y = +50;

   return Promise.resolve(mesh);
 }
\end{lstlisting}
\newpage

\subsection{Instanziazione nel catalogo}
Dopo aver implementato un Plugin è necessario al fine di poterlo inserire all'interno di una scena, instanziarlo
all'interno del \emph{Plugin Catalog}.\\
\begin{lstlisting}[
                   numbers=left,
                   frame=trBL,
                   basicstyle=\tiny,
                   caption={Instanziazione del Plugin nel Catalogo},
                   label=struttura]
import {Catalog} from 'react-planner';
...
import cube from './items/cubo/cubo';
...
catalog.registerElement(cube);
...
export default catalog;
\end{lstlisting}

\subsection{Testing nella scena}
La fine del processo porta al testing del nuovo \emph{Plugin} implementato, testandolo all'interno
di una scena per verificare che le caratteristiche, geometriche e dimensionali corrispondando con quelle
del modello reale dal quale si è preso spunto per la realizzazione del modello digitale.
