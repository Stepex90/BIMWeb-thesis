\section{Definizione}
\label{sec:chapter_3_section_1}

\emph{Plugin} \`e un componente software che pu\`o può essere perfettamente integrato nel sistema, al fine di estenderne le sue capacit\`a.
In \emph{Metior}, un plugin rappresenta un elemento architetturale che estende le Building Information Model progettate.
Tecnicamente, un plugin rappresenta un \emph{prototype} (RIVEDERE cioè una ``class'' in un Object Oriented Programming) di un elemento di
costruzione che può essere inserito (``instanziato'') nel \emph{canvas}, definendo cos\`i un nuovo \emph{elemento},
in altre parole un nuovo componente del modello.
\newpage

% \emph{plugin} is a software component that can be seamlessly integrated into the system in order to extends its capabilities.
% In \emph{Metior}, a plugin represents an architectural element that extends the Building Information Model design.
% Technically, a plugin represents a \emph{prototype} (namely a ``class'' in Object Oriented Programming) of a construction element
% that can be inserted (``instantiated'') into the \emph{canvas}, thus defining a new \emph{element}, i.e. a new component of the model.

\subsection*{Proprietà}

\noindent
Un plugin \`e descritto dalle seguenti otto propriet\`a:
\begin{itemize}
  \item un nome univoco;
  \item una descrizione;
  \item l'\emph{occupation type} (uno tra \emph{linear}, \emph{area} or \emph{volume});
  \item il \emph{placement type} (\emph{inside} or \emph{over});
  \item un insieme di proprietà specifiche che mappano la semantica da associare al plugin;
  \item  una \emph{generating function} che restituisce la rappresentazione 2D dell'elemento in formato SVG, da usare nel \emph{2D-mode};
  \item  una \emph{generating function} che restituisce la rappresentazione 3D dell'elemento in formato OBJ, da usare nel  \emph{3D-mode};
  \item un insieme di metadati che consente l'inserimento di informazioni generiche;
\end{itemize}

% A plugin is described by the following eight properties: (1) a unique name; (2) a description; (3) a set of metadata;
% (4) the \emph{occupation type} (one among \emph{linear}, \emph{area} or \emph{volume}); (5) the \emph{placement type} (\emph{inside} or \emph{over});
% (6) a set of specific properties mapping the semantic to associate to the plugin;
% (7) a \emph{generating function} that returns the 2D representation of the element in SVG format,to be used in the \emph{2D-mode};
% (8) a \emph{generating function} that returns the 3D representation of the element in OBJ format, to be used in the \emph{3D-mode}.
