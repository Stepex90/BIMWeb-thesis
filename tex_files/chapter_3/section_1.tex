\section{Definizione}
\label{sec:chapter_3_section_1}

\emph{Plugin} \`e un componente software che pu\`o può essere perfettamente integrato nel sistema, al fine di estenderne
le sue capacità.
In \emph{Metior}, un plugin rappresenta un elemento architetturale che estende le Building Information Model progettate.
Tecnicamente, un plugin rappresenta un \emph{prototype} di un elemento di
costruzione che può essere inserito (``instanziato'') nel \emph{canvas}, definendo cos\`i un nuovo \emph{elemento},
in altre parole un nuovo componente del modello.
\newpage

\subsection*{Proprietà}
\noindent
Un plugin \`e descritto dalle seguenti otto propriet\`a:
\begin{itemize}
  \item un nome univoco;
  \item una descrizione;
  \item l'\emph{occupation type} (uno tra \emph{linear}, \emph{area} or \emph{volume});
  \item il \emph{placement type} (\emph{inside} or \emph{over});
  \item un insieme di proprietà specifiche che mappano la semantica da associare al plugin;
  \item  una \emph{generating function} che restituisce la rappresentazione 2D dell'elemento in formato SVG, da usare nel \emph{2D-mode};
  \item  una \emph{generating function} che restituisce la rappresentazione 3D dell'elemento in formato OBJ, da usare nel  \emph{3D-mode};
  \item un insieme di metadati che consente l'inserimento di informazioni generiche;
\end{itemize}
