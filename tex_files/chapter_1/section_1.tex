\section{Building Information Modeling}
\label{sec:chapter_1_section_1}
\noindent

Una tendenza a minimizzare l'umanizzazione di nuovi territori e di spingere per il riutilizzo di alloggi
già costruiti e caduti in disuso è diventata una necessità pressante nelle società avanzate.
La direzione intrapresa è di integrare il modello ``zero energy'' (ogni costruzione è stata prodotta con il
consumo della stessa energia) con il modello ``zero waste'', nuovo paradigma di progettazione dove i materiali di
rifiuto della demolizione diventano risorse per ricostruire\cite{altamura:12}.
I processi di costruzione e progettazione necessitano di essere rinnovati per tener conto delle preoccupazioni
ambientali. Per ridurre l'impatto dei progetti di costruzione sull'ambiente, il progetto ha bisogno di prendere in
considerazione la questione dei materiali di costruzione.
Le amministrazioni pubbliche hanno bisogno di strumenti adeguati per il calcolo e il controllo dei materiali
riutilizzati o smaltiti.
I nuovi strumenti dovrebbero gestire l'elaborazione digitale dei materiali in tutto il ciclo di vita del progetto,
supportando i requisiti del nuovo processo come: progettazione per la Demolizione, Progettazione per il Riciclo e per i Rifiuti.
In particolare, un ciclo di vita dell'edificio è sostenuto da un processo di costruzione che prevede cicli allineati
con i fenomeni naturali.
In questo progetto di tesi si propongono delle  soluzioni che servono a chiudere il cerchio del ciclo di vita dell'edificio,
allontanandosi dalla tradizionale risposta lineare con tassi di energia ad alto consumo e verso il riutilizzo
dei materiali in decostruzione/ricostruzione, supportato da un processo computerizzato di demolizione selettiva.
Tutti i cicli di ristrutturazione degli edifici dovrebbero prevedere passi di de-costruzione e ri-costruzione, mirati
verso la sostituzione dei materiali al fine di ottenere una maggiore efficienza. Il trattamento di questi materiali
richiede la codifica appropriata sia per lo smaltimento, secondo i codici del CER (Catalogo Europeo dei Rifiuti),
sia per la pianificazione e la progettazione di nuovi edifici, seguendo la metodologia BIM (Building Information Modeling).
A questo scopo si necessita di scene georeferenziate di realtà aumentata sulla base di modelli 3D veloci,
facilmente navigabili e misurabili.
Si è a conoscenza dei costi di costruzione da zero, ma poco si sa circa i tassi di sostituzione di parti
di un edificio da realizzare tramite la demolizione selettiva. Un moderno processo di demolizione selettiva richiede l'intervento umano,
con costi assicurativi elevati a causa della pericolosità per coloro che lavorano in queste attività.
Quest'ultimo punto richiede un'alternativa alla sforzo umano in questi processi. \`E auspicabile che i robot automatici
sostituiscano il lavoro umano; ad esempio i droni potrebbero operare in un
contesto semanticamente familiare e dare aggiornamenti in tempo reale dell'ambiente rilevato.
C'è, quindi, bisogno di un moderno framework di facile utilizzo per la modellazione da usare per la
costruzione/decostruzione nel settore AEC (Architecture, Engineering and Construction),
per fornire un modello di realtà aumentata attraverso il riconoscimento semantico tramite computer vision e
fotogrammetrico con precisione centimetrica.\\
Tali strumenti di realtà virtuale/aumentata
richiedono sia una veloce modellazione della costruzione 3D e aumento del contenuto semantico, per poter essere controllata con
tempi quasi realtime: questa è la vera sfida richiesta anche dal futuro sviluppo con sistemi IoT.
\newpage
