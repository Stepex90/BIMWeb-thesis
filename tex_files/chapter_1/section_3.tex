
\section{Applicazioni BIM nel Web}
\label{sec:chapter_1_section_3}

Oggigiorno stiamo assistendo ad una migrazione inarrestabile di prodotti software verso i servizi accessibili
attraverso il mezzo web. Ciò è dovuto principalmente ai benefici innegabili in termini di accessibilità, usabilità,
manutenibilità e spalmabilità concesso dal mezzo Web stesso.
Tuttavia questi benefici non arrivano senza un costo: le prestazioni e lo sviluppo della complessità diventano
maggiori preoccupazioni nell'ambiente Web.
In particolare, a causa dell'introduzione di diversi livelli di astrazione non è sempre possibile "porta" desktop
applicazione nel regno Web, un aspetto da prendere in considerazione anche per le differenze rilevanti tra hardware
tutti i dispositivi dotati di un browser Web.
Può essere ancora più arduo di affrontare un' architettura di software distribuito(un client / server almeno uno)
indotta dalla piattaforma Web. Tuttavia sono sempre più ricche e complesse le applicazioni Web che sono apparite,
sostenute dalle API arricchite HTML5, che grazie al WebGL~\cite{webgl}
(Che consente l'accesso diretto alla GPU), Canvas~\cite{Munro:15:HCC} (API raster 2D) e SVG~\cite{Jackson:11:SVG}
(API di disegno vettoriale), ha aperto la strada per l'ingresso di applicazioni Web Graphic.
In questo lavoro riportiamo il nostro impegno verso la definizione di uno strumento di modellazione edifici basato sul Web
che supera le suddette difficoltà prestazionali e di sviluppo basandosi su un modello di progettazione del flusso di dati
unidirezionale e su un'architettura serverless rispettivamente.
Un'architettura serverless, al contrario di ciò che il nome può suggerire, in realtà si avvale di molti server specifici
diversi, funzionamento e la manutenzione di cui non le fanno onere per lo sviluppatore del progetto.
Questi diversi server possono essere visti come servizi di terze parti (tipicamente cloud-based) o funzioni eseguite
in contenitori effimeri (può durare solo per una invocazione) per gestire lo stato interno e la logica server-side.
L'interazione in realtime tra gli utenti che lavorano congiuntamente sullo stesso progetto di modellazione,
è ad esempio ottenuto tramite un terzo API parti per la collaborazione gli utenti remoti.
L'interfaccia utente strumento, interamente basata sul modello componenti Web, è stato mantenuto il più semplice possibile: è richiesto all'utente
 interagire principalmente con bidimensionali segnaposto simbolici rappresentano parti dell'edificio, evitando così complesso 3D
  interazioni.
La complessità del modello è quindi spostato dal modellatore per lo sviluppatore, che compila un estensibile \emph{catalog} di
\emph{bulding elements} personalizzabili. Il modellatore deve solo selezionare l'elemento richiesto, il luogo e parametrizzare
esso a seconda delle esigenze. È ovvio che un gran numero di elementi costruttivi deve essere fornito per garantire
l'adempimento delle maggior parte delle esigenze di modellazione.

% Nowadays we are seeing a relentless migration of software products toward services accessible via the Web medium.
% This is mainly due to the undeniable benefits in terms of accessibility, usability, maintainability and spreadability
% granted by the Web medium itself.
% Nevertheless these benefits don't come without a cost: performance  and development complexity become major concerns
% in the Web environment.
% In particular, due to the introduction of several abstraction layers it is not always feasible to "port" a desktop
% application into the Web realm, an aspect to be taken into account even for the relevant hardware differences among
%  all the devices equipped with a Web Browser.
% It can be even more arduous to tackle the inherent distributed software
%  architecture (a client/server one at least) induced by the Web platform. Nevertheless increasingly rich and complex
%   Web applications began to appear, supported by the enriched HTML5 APIs, which thanks to the WebGL~\cite{webgl}
%   (which enables direct access to GPU), Canvas~\cite{Munro:15:HCC} (2D raster APIs) and SVG~\cite{Jackson:11:SVG}
%    (vectorial drawing APIs), has paved the way for the entrance of Web Graphic Applications.
% In this work we report about our endeavor toward the definition of a Web based buildings modeling tool which overcomes
%  the aforementioned performance and development difficulties relying on a unidirectional data flow design pattern and
%  on a serverless architecture, respectively.
% A serverless architecture, on the contrary of what the name may suggest, actually employs many different specific servers,
%  whose operation and maintenance don't burden on the project developer(s). These several servers can be seen as third party
%   services (typically cloud-based) or functions executed into ephemeral containers (may only last for one invocation) to manage
%    the internal state and server-side logic. Realtime interaction among users jointly working on the same modeling project,
%    is for example achieved via a third party APIs for remote users collaboration.
% The tool user interface, entirely based on web components pattern,  has been kept as simple as possible: the user is required
%  to interact mainly with two-dimensional symbolic placeholders representing parts of the building, thus avoiding complex 3D
%   interactions.
% The modeling complexity is thus moved from the modeler to the developer which fills out an extendible \emph{catalog} of
% customizable \emph{building elements}. The modeler has only to select the required element, place and parametrize it
% according to the requirements. It is obvious that a large number of building elements has to be provided to ensure
% the fulfillment of the most modeling requirements.
% The remainder of this document is organized as follows. Section~\ref{sec:related_work} provides an overview of related work.
%  Section~\ref{sec:application} reports about the application user experience. Section~\ref{sec:architecture} presents adopted
%   architectural solutions. Finally, Section ~\ref{sec:conclusions} contains some conclusive remarks.
