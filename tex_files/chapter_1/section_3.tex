
\section{Tecnologie Web per il BIM}
\label{sec:chapter_1_section_3}

Il mondo dei prodotti software sta assistendo ad una migrazione inarrestabile verso i servizi accessibili
attraverso il mezzo Web. Ciò è dovuto principalmente ai benefici innegabili in termini di accessibilità, usabilità,
manutenibilità e diffusione concessi dal mezzo Web stesso.
Tuttavia questi benefici non arrivano senza un costo: le prestazioni e lo sviluppo della complessità diventano
maggiori preoccupazioni nell'ambiente Web.
In particolare, a causa dell'introduzione di diversi livelli di astrazione, non è sempre possibile ``portare'' le
applicazioni Desktop nel dominio Web, un aspetto da prendere in considerazione anche per le differenze rilevanti
tra gli hardware di tutti i dispositivi dotati di un browser Web.
Può essere ancora più arduo affrontare un' architettura di software distribuito (almeno un client/server)
indotta dalla piattaforma Web. Tuttavia sono sempre più ricche e complesse le applicazioni Web che sono comparse,
sostenute dalle API arricchite da HTML5, insieme a WebGL~\cite{webgl}
(che consente l'accesso diretto alla GPU), Canvas~\cite{Munro:15:HCC} (API raster 2D) e SVG~\cite{Jackson:11:SVG}
(API di disegno vettoriale), hanno aperto la strada per l'ingresso di applicazioni Web Graphic.
L'obiettivo è stato la definizione di uno strumento di modellazione di edifici basato sul Web
che supera le suddette difficoltà prestazionali e di sviluppo basandosi su un modello di progettazione del flusso di dati
unidirezionale e su un'architettura serverless.
Un'architettura serverless, al contrario di ciò che il nome può suggerire, in realtà si avvale di molti server specifici
diversi, togliendo l'onere del funzionamento e della manutenzione allo sviluppatore del progetto.
Questi diversi server possono essere visti come servizi di terze parti (tipicamente cloud-based) o funzioni eseguite
in contenitori effimeri (possono durare solo per una invocazione) per gestire lo stato interno e la logica server-side.
L'interazione in realtime tra gli utenti che lavorano congiuntamente sullo stesso progetto di modellazione
è ad esempio ottenuta tramite API di terze parti che rendono possibile la collaborazione tra utenti remoti.
L'interfaccia utente, interamente basata sul modello a componenti Web, è stata mantenuta il più semplice possibile:
è richiesto all'utente interagire principalmente con segnaposto simbolici bidimensionali, che rappresentano parti dell'edificio,
evitando così interazioni complesse in 3D.
La complessità del modello è quindi spostata dal modellatore allo sviluppatore, il quale deve compilare un \emph{catalogo} di
\emph{bulding elements} estendibile e personalizzabile. Il modellatore deve solo selezionare l'elemento richiesto,
il luogo e parametrizzare esso a seconda delle esigenze. È ovvio che un gran numero di building elements deve essere fornito
per garantire l'adempimento delle maggior parte delle esigenze di modellazione.
