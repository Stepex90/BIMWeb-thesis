Il lavoro presentato in questa tesi, svolto presso il \emph{CVDLAB},
dell'Università di Roma Tre, è consistito nello studio delle tecnologie e nello sviluppo del
framework \emph{Metior}.
Questo applicativo consente di modellare edifici in ambiente Web, secondo una metodologia che coniughi le potenzialità
dell’approccio BIM con la semplicità di utilizzo.
In particolar modo si è approfondita l'implementazione dei \emph{Plugin}, oggetti posizionabili all'interno della scena, creando
una procedura formalizzata del processo.
L'obiettivo principale è stato portare il concetto di BIM sul Web, al fine di modellare i diversi ambienti a seconda del
contesto in modo personalizzato.
Durante il percorso di tesi sono state di grande importanza, in termini di crescita personale e professionale,
le collaborazioni con \href{http://www.sogei.it/flex/cm/pages/ServeBLOB.php/L/IT/IDPagina/116}{SOGEI} (Società Generale d'Informatica S.p.A. la società IT del Ministero dell'Economia e delle Finanze)
e \href{http://www.cng.it/it/consiglio-nazionale}{CNG} (Comitato Nazionale Geometri).
La collaborazioni con SOGEI ha riguardato la modellazione di un \emph{Virtual CED} (Centro Elaborazione Dati),
realizzato con l'intento di avere un modello 3D navigabile, che consenta lo sviluppo di un sistema per
la generazione di Indoor Intercactive Virtual Mapping per il supporto ad applicazioni di Indoor-Location, Indoor-Navigation
ed IoT-Mapping.
L'altra collaborazione è stata quella con il \emph{CNG} per il progetto \emph{BaM}
(Bulding and Modelling),
facente parte del progetto nazionale \emph{Georientiamoci}, per l'orientamento degli alunni nella scelta del percorso di studi,
per il passaggio dalle scuole medie alle superiori.
Il progetto nasce con lo scopo di far realizzare agli alunni, ``\emph{l'aula che vorrei}''.
Questa iniziativa si prefigge di sensibilizzare gli studenti nell'utilizzo di
materiali ecosostenibili e rispettare l'ambiente.
Infine il progetto \emph{Deconstruction}, il quale nasce con l'idea di fare delle previsioni
sui costi di demolizione e smaltimento degli edifici, al fine di ottimizzare i processi e far
diventare questi materiali delle risorse per la ricostruzione.
