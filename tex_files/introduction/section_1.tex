Il lavoro presentato in questa tesi, svolto presso il \emph{CVDLAB},
laboratorio dell'\emph{Università di Roma Tre}, è consistito nello studio delle tecnologie e nello sviluppo del
framework \emph{Metior}.
Questo applicativo consente di modellare edifici in ambiente Web, secondo una metodologia che coniughi le potenzielità
dell’approccio BIM con la semplicità di utilizzo.
In particolar modo si è approfondita l'implementazione dei \emph{Plugins} ( oggetti posizionabili all'interno della scena ).
L' obiettivo principale è stato portare il concetto di BIM sul Web, al fine di modellare i diversi ambienti a seconda del
contesto.
Nello sviluppo del progetto sono state di grande importanza, in termini di crescita personale e professionale,
le collaborazioni con \emph{SOGEI} - (Società Generale d'Informatica S.p.A. una private company ICT)
nell'ambito della modellazione del CED (Centro Elaborazione Dati),
ed inoltre quella1 con il \emph{CNG} (Comitato Nazionale Geometri) per i progetti BaM (Bulding and Modelling) e Deconstruction.
