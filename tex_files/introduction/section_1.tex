Il lavoro descritto in questa tesi si pone proprio in questo contesto, in un ambiente Web abilitante il BIM.
Il lavoro è stato svolto nel periodo da Ottobre 2016 a Marzo 2017, presso il \emph{CVDLAB} - (Computational Visual Design Lab)
dell'Università di Roma Tre, in collaborazione con \emph{SOGEI} e \emph{GeoWeb}
ed è consistito nello studio delle tecnologie e nello sviluppo del framework \emph{Metior}.
Questo applicativo consente di modellare edifici in ambiente Web, secondo una metodologia che coniughi le potenzialità
dell’approccio BIM con la semplicità di utilizzo.
In particolar modo si è approfondita l'implementazione dei \emph{Plugin}, oggetti posizionabili all'interno della scena, creando
una procedura formalizzata del processo.
L'obiettivo principale è stato portare il concetto di BIM sul Web, al fine di modellare i diversi ambienti a seconda del
contesto in modo personalizzato.
Durante il percorso di tesi sono state di grande importanza, in termini di crescita personale e professionale,
le collaborazioni con \href{http://www.sogei.it/flex/cm/pages/ServeBLOB.php/L/IT/IDPagina/116}{\emph{SOGEI}}
(Società Generale d'Informatica S.p.A. la società IT del Ministero dell'Economia e delle Finanze)
e \href{http://www.geoweb.it/}{\emph{GeoWeb}}.
La collaborazione con \emph{SOGEI} ha riguardato la modellazione di un \emph{Virtual CED} (Centro Elaborazione Dati),
realizzato con l'intento di avere un modello 3D navigabile, che consenta lo sviluppo di un sistema per
la generazione di Indoor Interactive Virtual Mapping per il supporto ad applicazioni di Indoor-Location, Indoor-Navigation
ed IoT-Mapping.
La collaborazione con \emph{GeoWeb} ha portato alla realizzazione di due progetti: il primo \emph{BaM} (Building and Modelling),
facente parte del progetto nazionale \emph{Georientiamoci},
sponsorizzato dal \href{http://www.cng.it/it/consiglio-nazionale}{CNG} (Comitato Nazionale Geometri)
per l'orientamento degli alunni nella scelta del percorso di studi nel passaggio dalle scuole medie alle superiori;
il secondo progetto \emph{Deconstruction}, invece, nasce dalla necessità di fare delle previsioni
sui costi di demolizione e smaltimento degli edifici, al fine di ottimizzare i processi e far
diventare questi materiali delle risorse per la ricostruzione.
\newpage
