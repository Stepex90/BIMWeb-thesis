Nella tesi gli argomenti trattati, sono stati sviluppati in quattro capitoli.
Nel primo capitolo si descrive il concetto di BIM, nell’ambito della modellazione propondendo una possibile soluzione
per portare questa metodologia in modo ``semplicato'' sulle piattaforme Web.
\`E stato fatto uno studio sullo stato dell’arte, per fornire un overview
sugli applicativi Desktop disponibili sul mercato. Si descrive inoltre come è
possibile portare la modellazione sulle piattaforme Web e attraverso quali scelte tecnologiche.
Nel secondo capitolo si introduce la scelta fatta per portare il BIM sul Web, introducendo \emph{Metior}.
Il framework implementa le funzionalità del BIM attraverso l'utilizzo di librerie software,
in particolar modo Reactjs~\ref{sec:chapter_2_section_3_sub_1} e Threejs ~\ref{sec:chapter_2_section_3_sub_2}.
L'utente dispone di un applicativo paragonabile a quelli
Desktop come potenzialità, ma con una maggiore facilità di utilizzo, consentendo l'interazione del modello creato
tramite una visualizzazione 2D e 3D.
Nel terzo capitolo si descrive come vengono implementati i plugins utilizzati all'interno di Metior,
dando una descrizione completa delle caratteristiche intrinseche dei modelli. L'implementazione ha l'obiettivo
di dare all'utente un esperienza soddisfacente attraverso la personalizzazione dei plugins a seconda del contesto.
Nel quarto capitolo si descrivono i contesti di applicazione, facendo un overview sul progetto BaM~\ref{sec:chapter_4_section_1}
realizzato in collaborazione con il CNG e la modellazione di un Virtual CED~\ref{sec:chapter_4_section_2} all'interno di SOGEI, ed
infine il progetto Deconstruction~\ref{sec:chapter_4_section_3}.
