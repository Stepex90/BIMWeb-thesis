L'elaborato si articola in cinque capitoli.
Nel primo capitolo si descrive il concetto di BIM, nell’ambito della modellazione, proponendo una possibile soluzione
per portare questa metodologia in modo ``semplificato'' sulle piattaforme Web.
Si presenta uno studio sullo stato dell’arte per fornire un overview
sugli applicativi Desktop disponibili sul mercato. Si descrive, inoltre, come è
possibile portare la modellazione sulle piattaforme Web e attraverso quali scelte tecnologiche.\\
Nel secondo capitolo si introduce la scelta fatta per portare il BIM sul Web, introducendo \emph{Metior}.
Il framework implementa le funzionalità del BIM attraverso l'utilizzo di librerie software open-source,
in particolar modo \emph{React}~\ref{sec:chapter_2_section_3_sub_1} e \emph{Threejs} ~\ref{sec:chapter_2_section_3_sub_2}.
L'utente dispone di un applicativo paragonabile a quelli
Desktop come potenzialità, ma con una maggiore facilità di utilizzo, consentendo l'interazione del modello creato
tramite una visualizzazione 2D e 3D.\\
Nel terzo capitolo si fornisce una descrizione completa di un \emph{Plugin} delle
proprietà caratteristiche e la loro tassonomia ad alto livello, fino a definire in modo formalizzato
la procedura implementativa all'interno del framework \emph{Metior}.\\
Nel quarto capitolo si descrivono i contesti di applicazione, facendo un overview sul
progetto \emph{BaM}~\ref{sec:chapter_4_section_1} realizzato in collaborazione con \emph{GeoWeb} per il CNG,
ed il progetto \emph{Deconstruction}~\ref{sec:chapter_4_section_2}.
L'altra collaborazione con \emph{SOGEI} per la modellazione di un \emph{Virtual CED}~\ref{sec:chapter_4_section_3}.
