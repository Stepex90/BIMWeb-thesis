Nella tesi gli argomenti trattati, sono stati sviluppati in quattro capitoli.
Nel primo capitolo si descrive il concetto di BIM,
nell'ambito della modellazione 3D su piattaforme Web. \`E stato fatto uno studio sullo stato dell'arte
per fornire un overview sugli applicativi Desktop disponibili sul mercato.
Si descrive inoltre come \`e possibile portare la modellazione sulle piattaforme Web.
Nel secondo capitolo si introduce la scelta fatta per portare il BIM sul Web, proponendo una nostra soluzione Metior.
Il framework implementa le funzionalit\`a del BIM attraverso l'utilizzo di librerie software,
in particolar modo Reactjs~\ref{sec:chapter_2_section_3_sub_1} e Threejs ~\ref{sec:chapter_2_section_3_sub_2} ,
dando all'utente un applicativo paragonabile a quelli
Desktop, che consente l'interazione del modello creato tramite una visualizzazione 2D e 3D.
Nel terzo capitolo si descrive come vengono implementati i plugins utilizzati all'interno di Metior,
dando una descrizione completa delle caratteristiche intrinseche dei modelli.
Nel quarto capitolo si descrivono i contesti di applicazione del progetto sviluppato,
facendo un overview sui progetti BaM~\ref{sec:chapter_4_section_1}
 e Deconstruction~\ref{sec:chapter_4_section_2} realizzati in collaborazione con il CNG
e la modellazione del CED~\ref{sec:chapter_4_section_3} all'interno di SOGEI.
