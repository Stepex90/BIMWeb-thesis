\section{Conclusioni}
\label{sec:conclusions_section_1}

In questa tesi si è contribuito ad implementare un'architettura serverless per supportare la modellazione di edifici
in ambiente Web. L'architettura serverless ha dato dei benifici in termini di disponibilità, affidabilità,  scalabilità,
facilità di implementazione, manutenzione e aggiornabilità ottenutasi implementando un applicazione logica
come client-side con un solo stato centralizzato (nella forma di un documento JSON) che può essere messo in un
document oriented database anche di terze parti e caricato nell'architettura frontend, la quale
 trasparentemente ricarica lo stato nella sua versione serializzata che gli è stato passato .
 L'applicazione stessa può essere servita da un CDN (Content Delivery Network) evitando così la necessità di un server web.
(le parti lato server rendering e gestione condivisa utenti può essere demanda su cloud)
 % La routine non in linea si basa su una piattaforma Function-as-a-Service così come le caratteristiche
 % di gestione utenti e collaborazione.
