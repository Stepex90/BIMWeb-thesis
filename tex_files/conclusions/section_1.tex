\section{Conclusioni}
\label{sec:conclusions_section_1}

Il lavoro presentato in questa tesi ha contribuito allo sviluppo del framework \emph{Metior}
per supportare la modellazione di edifici in ambiente Web ed in particolar modo l'implementazione dei \emph{Plugin}
personalizzandoli a seconda del caso d'uso reale.
Nel framework l'architettura \emph{serverless} ha dato dei benefici
in termini di disponibilità, affidabilità, scalabilità, facilità di implementazione, manutenzione e aggiornabilità,
ottenutasi implementando un'applicazione logica client-side con un solo stato centralizzato nella forma di un documento JSON.
Tale stato dell'applicazione può essere messo in un
document oriented database anche di terze parti e caricato nell'architettura frontend, la quale
trasparentemente lo ricarica nella sua versione serializzata.
L'applicazione stessa può essere servita da un CDN (Content Delivery Network) evitando così la necessità di un server web.
