\section{Conclusioni}
\label{sec:conclusions_section_1}

In questa tesi si è contribuito ad implementare un'architettura serverless per supportare la modellazione di edifici
in ambiente Web. L'architettura serverless ha dato dei benifici in termini di disponibilità, affidabilità,  scalabilità,
facilità di implementazione, manutenzione e aggiornabilità si è ottenuta implementando un applicazione logica
come client-side con un solo stato centralizzato (nella forma di un documento JSON) che può essere messo in
document oriented DB-as-a-Service di terze parti e caricato nell'architettura frontend, la quale
 trasparentemente ricarica lo stato nella sua versione serializzata che gli è stato passato .
 L'applicazione stessa è servita da un CDN (Content Delivery Network) evitando così la necessità di un server web.
 La routine non in linea si basa su una piattaforma Function-as-a-Service così come le caratteristiche
 di gestione utenti e collaborazione.

% In this work we outlined a serverless architecture to support buildings modeling in a Web environment.
%  The serverless  architecture that gives benefits in terms of availability, reliability, scalability,
%  easiness of deployment, maintainability and upgradability is obtained by implementing the application
%   logic as a client-side only centralized state Web application exploiting the unidirectional data flow pattern.
%    This approach allows for a easy-to-serialize state (in the form of a JSON document) that can be pushed on a
%    third party document oriented DB-as-a-Service and loaded back in the frontend reactive architecture, which
%    transparently reload the state once its serialized version is passed in. The application itself is served by
%     a CDN (Content Delivery Network) thus avoiding any need for web server. Offline routines rely on Function-as-a-Service
%      platform as well as users management and collaboration features.
