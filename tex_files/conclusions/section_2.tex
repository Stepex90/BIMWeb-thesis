\section{Sviluppi futuri}
\label{sec:conclusions_section_2}

I possibili sviluppi e campi di utilizzo del framework implementato possono essere tanti.
Si \`e deciso di seguire principalmente tre strade di sviluppo:
\begin{itemize}
\item Integrazione IoT
\item Collaboratività
\item Fotorealismo
\end{itemize}
\newpage

\subsection{IoT}
Nel mondo gli oggetti presenti nella vita di tutti i giorni, hanno al proprio interno del hardware che consente di interagire
con essi. Questi dispositivi sono denominati IoT (Internet of Things) e consentono l'estensione su Internet degli oggetti e dei
luoghi reali. Questo ci consente di pensare all'inserimento di meta dati all'interno dei plugin implementati,
consentendo all'utente un interazione realtime consentendo la fruizione di informazioni
intriseche al modello.

%inserire foto di un plugin con sprite
\newpage

\subsection{Collaboratività}

Un altro possibile sviluppo del framework \'e inserire la modalità \emph{Collaborativa}, consentendo a più utenti
di collaborare su uno stesso progetto, ottimizando i tempi di lavoro. Si fa riferimento
all'utilizzo all'interno di uno studio di geometri. Lo stack tecnologico scelto per questo sviluppo sono
\emph{Firebase} e \emph{jsondiff}. Il primo \'e necessario, per consentire in un possibile sistema di autenticazione degli utenti,
fare un controllo degli accessi e dei permessi al progetto.
Il secondo modulo presente in npm, consente di confrontare nel contesto collaborativo i file JSON su cui lavorono
gli utenti per trovare le modifiche apportate.

% inserire screenshoot

\newpage

\subsection{Fotorealismo}
Con fotorealismo si intende semplicemente che una scena simulata \`e indistinguibile da una fotografia, o per estensione
dalla vita di tutti i giorni. Questo \'e possibile attraverso la \emph{rasterizzazione}, che consiste in un algoritmo che
permette di convertire un'immagine a due dimensioni in una formata da pixel per avere fotogrammi proiettabili sugli schermi.

% \subsubsection{Baking Service}
Il servizio \emph{Baking} \`e un servizio remoto che prende una rappresentazione della scena in JSON come input,
calcola le texture lightmapped, le Mappe Ambientali per riflessione e rifrazione e memorizza le informazioni grafiche
in un formato che \`e compatibile con quello d'ingresso, in modo da permettere l'anello di retroazione di authoring.

% The Baking Service is a remote web service that takes a JSON scene representation as its input,
% computes the lightmapped textures, the envmaps for reflections and refractions and stores the enhanced graphic information
%  in a format that is compliant with the input, so enabling the feedback authoring loop.

Lo scopo principale \`e semplificare il workflow del visual 3D per fornire una user-experience ad alto livello
sul cliente web (desktop, tablet e mobile, wearables come Google Carboard ).
Compattare le strutture dati 3D prodotte dall'editor sul browser, trasmetterle ad un servizio di bakin web remoto,
e restituire una piacevole esperienza VR in real-time con alto realismo e frame rate.
Attualmente stiamo fornendo diverse ottimizzazioni, tra cui una soluzione per far fronte a scene di memoria basata
 su portali e frammentazione del ambienti caricando su richiesta solo una parte degli interi dati generati.
Il progetto discusso fa parte di un programma di grande che fornisce Indoor mapping e indoor/outdoor 3D di modelli realistici
partendo da (a) documenti catastali e/o disegni di costruzione, e (b) outdoor/indoor voli con drone che restituiscono
un set di fotografie e di nuvole di punti generati.
Il possibile range di applicazioni in cui utilizzare il framework spazia dalla sicurezza all'interno di piccole aree o edifici pubblici, a e-commerce,
accesso virtuale al patrimonio culturale, ai videogames, e molto altro ancora.

% In this paper we have discussed a simplified visual 3D workflow to provide a high-level user-experience
%  on web clients (desktop, tablet and mobile, reaching wearables like Google Cardboard).
%  Compact 3D data structures are produced by the editor on the browser, transmitted to a remote web baking service,
%   and given back as a real-time enjoyable VR experience with high realism and frame rate. We are currently providing
%   several optimizations including a solution to cope with out of memory scenes based on portals and fragmentation of
%   the environments, thus loading on demand only a portion of the whole generated data.

 % We are also working to automatize the generation of built environments using the novel LAR (Linear Algebraic Representation)
 %   data structure~\cite{Dicarlo:2014:TNL:2543138.2543294}.

% The project discussed here is actually a component of a much larger programme to provide indoor mapping and indoor/outdoor 3D
%  realistic models starting from (a) cadaster documents and/or building drawings, and (b)  outdoor/indoor drone flights and their
%   returned set of photographs and generated point clouds.
% The possible applications range from security enforcing inside small areas and public buildings, to e-commerce, to virtual access
%  to cultural heritage, to serious games, and much more.

\newpage
