\section{Sviluppi futuri}
\label{sec:conclusions_section_2}

I possibili sviluppi e campi di utilizzo del framework \emph{Metior} implementato possono essere tanti.
Dopo aver utilizzato in scenari reali il framework, si è ritenuto di focalizzare gli sviluppi successivi nei seguenti ambiti:
(a) integrazione con dispositivi IoT, (b) progettazione collaborativa, (c) fotorealismo.

\subsection{IoT}
\label{sec:conclusions_section_2_sub_1}
Nel mondo gli oggetti presenti nella vita di tutti i giorni contengono del hardware e del software
che consentono di interagire con essi da remoto, questi dispositvi grazie all'evoluzione tecnologica sono chiamati
a comunicare in una forma sempre più interconnessa. Questi dispositivi sono denominati \emph{IoT} (Internet of Things) e
consentono l'estensione su Internet degli oggetti e dei luoghi reali. Questo consente di pensare all'inserimento di
metadati all'interno dei \emph{Plugin} implementati nel framework \emph{Metior},
consentendo all'utente un interazione realtime con i dispositivi e le loro informazioni.\\

\begin{figure}[htbp] %  figure placement: here, top, bottom, or page
   \centering
   \includegraphics[width=1\linewidth]{images/iot}
   \caption{Plugin con informazioni IoT visualizzate tramite sprite}
   \label{fig:iot}
   \end{figure}

\newpage

\subsection{Progettazione Collaborativa}
\label{sec:conclusions_section_2_sub_2}
Un altro possibile sviluppo del framework è inserire la modalità di \emph{Progettazione Collaborativa}, consentendo a più utenti
di collaborare contemporaneamente su uno stesso progetto, ottimizando i tempi di lavoro.
Lo stack tecnologico scelto per questo sviluppo sono
\emph{Firebase} e \emph{jsondiff} modulo di npm.\\
\emph{Firebase}~\cite{firebase} è una piattaforma mobile e per applicazioni web con strumenti e infrastrutture progettati
per aiutare gli sviluppatori a creare applicazioni di alta qualità. Si è scelto di utilizzare
il Database Realtime di Firebase  il quale è un database cloud-hosted. I dati vengono memorizzati come JSON e
sincronizzati in tempo reale ad ogni client connesso. Quando si crea applicazioni cross-platform con
tecnologie Android, iOS SDK, e JavaScript, tutti i client condividono una istanza in tempo reale del
database e ricevere automaticamente gli aggiornamenti con i dati più recenti.
Esso è stato scelto per implementare un possibile sistema di autenticazione degli utenti,
fare un controllo degli accessi e dei permessi sul progetto,
essendo che il framework salva lo stato dell'applicazione in un file JSON.\\
Si è deciso inoltre di utilizzare il gestore di pacchetti e moduli \emph{npm}~\cite{npm} il quale rende facile per gli sviluppatori
JavaScript condividere e riutilizzare il codice e rendendolo facile da aggiornare.
Un \emph{pacchetto} è un file o una directory con uno o più file in essa contenuti, che viene descritto da un "package.json"
con alcuni metadati su di esso. Una tipica applicazione, ad esempio un sito web, dipenderà da decine o centinaia di pacchetti.
Questi pacchetti sono spesso piccoli. L'idea generale è che si crea un piccolo blocco di istruzioni che risolve un problema.
In questo modo è possibile risolvere un problema più grande, attraverso una soluzione personalizzata di questi piccoli,
blocchi condivisi.
\newpage

Si è scelto il modulo \emph{jsondiff} fornito da \emph{npm}, il quale consente di confrontare nel contesto collaborativo
i file JSON su cui lavorono gli utenti per trovare le modifiche apportate.\\

\noindent
\begin{minipage}{.45\textwidth}
\begin{lstlisting}[
                   numbers=left,
                   frame=trBL,
                   basicstyle=\tiny,
                   caption={JSON prima della modifica},
                   label=struttura]
  {
    "id": "HJAe59YF8Ux",
    "x": 201,
    "y": 891,
    "prototype": "vertices",
    "selected": false,
    "lines": ["Hype99FK88x", "S1w-hqKKL8e"],
    "areas": ["Byg1oFKUIe"]
  }
\end{lstlisting}
\end{minipage}\hfill
\begin{minipage}{.45\textwidth}
\begin{lstlisting}[
                   numbers=left,
                   frame=trBL,
                   basicstyle=\tiny,
                   caption={JSON dopo la modifica},
                   label=struttura]
 {
   "id": "HJAe59YF8Ux",
   "x": 201,
   "y": 891,
   "prototype": "vertices",
   "selected": false,
   "lines": ["Hype99FK88x", "S1w-hqKKL8e"],
   "areas": ["Byg1oFKUIe"]
 }
\end{lstlisting}
\end{minipage}

Sfruttando le tecnologie sopra descrite, si può pensare di sviluppare il framework \emph{Metior}
per la Progettazione Collaborativa pensando di gestire le modifiche apportate dai diversi utenti che lavorano su uno
stesso progetto seguendo un workflow simile a quello che usa GitHub nella gestione dei repository,
consentendo all'utente di effettuare operazioni come merge, push e pull.
Per gestire il problema della conflittualità sulle modifiche apportate dagli utenti,
si potrebbe inserire una gestione intelligente dei layer consentendo all'utente di fare un ``lock'' sul layer sul quale
sta lavorando non consentendo cosi agli altri utenti di apportare modifiche fino al suo rilascio.
\newpage

\subsection{Fotorealismo}
\label{sec:conclusions_section_2_sub_3}
Con \emph{Fotorealismo} si intende semplicemente che una scena simulata \`e indistinguibile da una fotografia, o per estensione
dalla vita di tutti i giorni. Questo è possibile attraverso la \emph{rasterizzazione}, che consiste in un algoritmo che
permette di convertire un'immagine a due dimensioni in una formata da pixel per avere fotogrammi proiettabili sugli schermi.

Il servizio \emph{Baking}~\cite{baking} \`e un servizio remoto che prende una rappresentazione della scena in JSON come input,
calcola le texture lightmapped, le Mappe Ambientali per riflessione e rifrazione e memorizza le informazioni grafiche
in un formato che \`e compatibile con quello d'ingresso, in modo da permettere l'anello di retroazione di authoring(che significa?).\\

\begin{figure}[htbp] %  figure placement: here, top, bottom, or page
   \centering
   \includegraphics[width=1\linewidth]{images/explorer-a-1}
   \caption{Esempio di un ambiente realizzato con il fotorealismo}
   \label{fig:revit}
   \end{figure}

Lo scopo principale \`e semplificare il workflow durante la visualizzazione 3D per fornire una \emph{User Experience}
ad alto livello sul browser (desktop, tablet e mobile) o su \emph{wearables} come Google Carboard.
Compattare le strutture dati 3D prodotte dall'editor sul browser, trasmetterle ad un servizio di baking web remoto,
e restituire una piacevole esperienza VR in real-time con alto realismo e frame rate.

\begin{figure}[htbp] %  figure placement: here, top, bottom, or page
   \centering
   \includegraphics[width=1\linewidth]{images/vr}
   \caption{Esempio di visione di un ambiente attraverso VR}
   \label{fig:revit}
   \end{figure}

Il fotorealismo può essere esteso in un contesto che fornisce Indoor mapping e indoor/outdoor 3D di modelli realistici
partendo da (a) documenti catastali e/o disegni di costruzione, e (b) scansioni outdoor/indoor realizzate tramite droni
che restituiscono un set di fotografie e di nuvole di punti.
Il possibile range di applicazioni in cui utilizzare il framework spazia dalla sicurezza all'interno di piccole aree
o edifici pubblici, a e-commerce, accesso virtuale al patrimonio culturale, ai videogames, e molti altri ancora.
\newpage
