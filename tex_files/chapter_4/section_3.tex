\section{Deconstruction}
\label{sec:chapter_4_section_3}

Il progetto \emph{Deconstruction} nasce con l'idea di fare delle previsioni
sui costi di demolizione e smaltimento dei materiali di scarto degli edifici (Figura ~\ref{fig:augmented}).
Il progetto ha lo scopo di promuovere l'uso di strumenti informatici semplificati per sostenere la decostruzione.
In particolare, fornisce un modellazione geometrica semplificata dell'edificio permettendo l'integrazione di una descrizione
semantica delle componenti e dei loro materiali.
La realtà Virtuale/Aumentata aiuta a superare le difficoltà amministrative, a condizione di avere una
corretta identificazione dei rifiuti prodotti. Questo approccio aumenta l'adozione di comportamenti virtuosi,
cioè il recupero e riutilizzo.
In particolare, una modellazione geometrica dell'edificio permette di individuare:
\begin{itemize}
  \item  costi/entrate derivanti da alternative di riciclo/riutilizzo, invece di smaltimento;
  \item  la composizione e l'integrazione di informazioni utili per la pianificazione delle attività di costruzione;
  \item  realizzazione delle soglie di riutilizzo/recupero previsti dalla normativa;
  \item  capacità di confrontare economicamente diverse opzioni.
\end{itemize}

Il progetto ha avuto inizio prendendo in considerazione il sistema SMARTWaste~\cite{smartWaste}.
Questo approccio permette di ricavare stime delle quantità di materiali, fornendo una descrizione del tipo di edificio
e la zona in cui è stato costruito. Con queste informazioni, si fornisce una rappresentazione aggregata dei dati di
interesse sono riempiendo automaticamente delle form.
Il framework \emph{Metior} nel contesto della decostruzione al contrario fornisce sia una modellazione geometrica di sottosistemi
e componenti edilizi e un annotazione semantica con materiali da costruzione, come una sorta di \emph{BIM semplificato}.
È un dato di fatto, che il settore edile nazionale è fortemente eterogeneo, necessita di una modellazione
dettagliata per ottenere delle informazioni sufficientemente accurate.
Un punto di vista di questo approccio è il carattere iterativo incrementale, in cui ogni fase di modellazione può essere
seguita da una validazione dei costi parziali.

\begin{figure}[htbp] %  figure placement: here, top, bottom, or page
   \centering
   \includegraphics[width=1\linewidth]{images/3d-sel}
   \caption{Vista 3D di un modello per il Deconstruction}
   \label{fig:augmented}
\end{figure}

Un rapporto completo sull'utilizzo BIM come un approccio edificio decostruzione è fornita da~\cite{galic2014bim}.
Uno studio sull'uso del BIM come supporto per la progettazione per Deconstruction è svolta da~\cite{akinade2015waste}.
In questa configurazione, al contrario di ciò di quella usata nel framework \emph{Metior}, la decostruzione deve essere
presa in considerazione a partire dall'inizio della progettazione degli edifici.
\newpage


Il framework \emph{Metior} (dal latino: a \emph{misura} o \emph{stima}),
ha come obiettivo superare tali difficoltà, tramite:
\begin{itemize}
  \item la progettazione e realizzazione di un \emph{web service} fornendo un'interfaccia utente semplificata;
  \item memorizzare un database crescente di \emph{Plugin} che rappresentano un modello per le parti di edificio geometricamente più complesse;
  \item utilizzando un motore geometrico estensibile e un Server basato su decenni di ricerca;
  \item offrendo integrazioni semantiche flessibili attraverso la specializzazione di (Industry Foundation Classes~\cite{ifc})
        IFC classi associate ai sottosistemi di costruzione e le parti;
\end{itemize}


Considerando ad esempio dei grandi siti che devo subire il processo di decostruzione
sono gestiti dai imprenditori, i quali hanno a disposizione e già utilizzato le competenze e strumenti specifici,
per la maggiorparte delle attività di decostruzione. La maggiorparte dei materiali di scarto prodotto,
sono gestiti da geometri provenienti da aziende di piccole o medie dimensioni o anche da singoli professionisti.
Questo tipo di società necessitano di un sostegno dato da strumenti in cui le complessità,
sia burocratiche che tecniche, devono essere nascoste, anche se correttamente gestite.
